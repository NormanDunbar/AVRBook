%%%%%%%%%%%%%%%%%%%%%%%%%%%%%%%%%%%%%%%%%
% The Legrand Orange Book
% Structural Definitions File
% Version 2.0 (9/2/15)
%
% Original author:
% Mathias Legrand (legrand.mathias@gmail.com) with modifications by:
% Vel (vel@latextemplates.com)
% 
% This file has been downloaded from:
% http://www.LaTeXTemplates.com
%
% License:
% CC BY-NC-SA 3.0 (http://creativecommons.org/licenses/by-nc-sa/3.0/)
%
%%%%%%%%%%%%%%%%%%%%%%%%%%%%%%%%%%%%%%%%%

%----------------------------------------------------------------------------------------
%	VARIOUS REQUIRED PACKAGES AND CONFIGURATIONS
%----------------------------------------------------------------------------------------

\usepackage[top=3cm,bottom=3cm,left=3cm,right=3cm,headsep=10pt,a4paper]{geometry} % Page margins

\usepackage{graphicx} % Required for including pictures
\graphicspath{{Pictures/}} % Specifies the directory where pictures are stored

\usepackage{lipsum} % Inserts dummy text

\usepackage{tikz} % Required for drawing custom shapes

\usepackage[english]{babel} % English language/hyphenation

\usepackage{enumitem} % Customize lists
\setlist{nolistsep} % Reduce spacing between bullet points and numbered lists

\usepackage{booktabs} % Required for nicer horizontal rules in tables

\usepackage{xcolor} % Required for specifying colors by name

%----------------------------------------------------------------------------------------
% NDunbar:
% Pull in my definitions file. All colour names are as per
% https://www.w3schools.com/colors/colors_names.asp but prefixed with www in case
% of clashes.
%
% NDunbar:. Because I can't get colours to work!
% They are supposed to be part of the xcolor package, but ....
%----------------------------------------------------------------------------------------

%----------------------------------------------------------
% Norman Dunbar
% A list of all the standard HTML colours as per
% https://www.w3schools.com/colors/colors_names.asp
% For my own use with the xcolor package.
%----------------------------------------------------------

\definecolor{wwwAliceBlue}{HTML}{F0F8FF}
\definecolor{wwwAntiqueWhite}{HTML}{FAEBD7}
\definecolor{wwwAqua}{HTML}{00FFFF}
\definecolor{wwwAquamarine}{HTML}{7FFFD4}
\definecolor{wwwAzure}{HTML}{F0FFFF}
\definecolor{wwwBeige}{HTML}{F5F5DC}
\definecolor{wwwBisque}{HTML}{FFE4C4}
\definecolor{wwwBlack}{HTML}{000000}
\definecolor{wwwBlanchedAlmond}{HTML}{FFEBCD}
\definecolor{wwwBlue}{HTML}{0000FF}
\definecolor{wwwBlueViolet}{HTML}{8A2BE2}
\definecolor{wwwBrown}{HTML}{A52A2A}
\definecolor{wwwBurlyWood}{HTML}{DEB887}
\definecolor{wwwCadetBlue}{HTML}{5F9EA0}
\definecolor{wwwChartreuse}{HTML}{7FFF00}
\definecolor{wwwChocolate}{HTML}{D2691E}
\definecolor{wwwCoral}{HTML}{FF7F50}
\definecolor{wwwCornflowerBlue}{HTML}{6495ED}
\definecolor{wwwCornsilk}{HTML}{FFF8DC}
\definecolor{wwwCrimson}{HTML}{DC143C}
\definecolor{wwwCyan}{HTML}{00FFFF}
\definecolor{wwwDarkBlue}{HTML}{00008B}
\definecolor{wwwDarkCyan}{HTML}{008B8B}
\definecolor{wwwDarkGoldenRod}{HTML}{B8860B}
\definecolor{wwwDarkGray}{HTML}{A9A9A9}
\definecolor{wwwDarkGreen}{HTML}{006400}
\definecolor{wwwDarkKhaki}{HTML}{BDB76B}
\definecolor{wwwDarkMagenta}{HTML}{8B008B}
\definecolor{wwwDarkOliveGreen}{HTML}{556B2F}
\definecolor{wwwDarkOrange}{HTML}{FF8C00}
\definecolor{wwwDarkOrchid}{HTML}{9932CC}
\definecolor{wwwDarkRed}{HTML}{8B0000}
\definecolor{wwwDarkSalmon}{HTML}{E9967A}
\definecolor{wwwDarkSeaGreen}{HTML}{8FBC8F}
\definecolor{wwwDarkSlateBlue}{HTML}{483D8B}
\definecolor{wwwDarkSlateGray}{HTML}{2F4F4F}
\definecolor{wwwDarkTurquoise}{HTML}{00CED1}
\definecolor{wwwDarkViolet}{HTML}{9400D3}
\definecolor{wwwDeepPink}{HTML}{FF1493}
\definecolor{wwwDeepSkyBlue}{HTML}{00BFFF}
\definecolor{wwwDimGray}{HTML}{696969}
\definecolor{wwwDodgerBlue}{HTML}{1E90FF}
\definecolor{wwwFireBrick}{HTML}{B22222}
\definecolor{wwwFloralWhite}{HTML}{FFFAF0}
\definecolor{wwwForestGreen}{HTML}{228B22}
\definecolor{wwwFuchsia}{HTML}{FF00FF}
\definecolor{wwwGainsboro}{HTML}{DCDCDC}
\definecolor{wwwGhostWhite}{HTML}{F8F8FF}
\definecolor{wwwGold}{HTML}{FFD700}
\definecolor{wwwGoldenRod}{HTML}{DAA520}
\definecolor{wwwGray}{HTML}{808080}
\definecolor{wwwGreen}{HTML}{008000}
\definecolor{wwwGreenYellow}{HTML}{ADFF2F}
\definecolor{wwwHoneyDew}{HTML}{F0FFF0}
\definecolor{wwwHotPink}{HTML}{FF69B4}
\definecolor{wwwIndianRed }{HTML}{CD5C5C}
\definecolor{wwwIndigo  }{HTML}{4B0082}
\definecolor{wwwIvory}{HTML}{FFFFF0}
\definecolor{wwwKhaki}{HTML}{F0E68C}
\definecolor{wwwLavender}{HTML}{E6E6FA}
\definecolor{wwwLavenderBlush}{HTML}{FFF0F5}
\definecolor{wwwLawnGreen}{HTML}{7CFC00}
\definecolor{wwwLemonChiffon}{HTML}{FFFACD}
\definecolor{wwwLightBlue}{HTML}{ADD8E6}
\definecolor{wwwLightCoral}{HTML}{F08080}
\definecolor{wwwLightCyan}{HTML}{E0FFFF}
\definecolor{wwwLightGoldenRodYellow}{HTML}{FAFAD2}
\definecolor{wwwLightGray}{HTML}{D3D3D3}
\definecolor{wwwLightGreen}{HTML}{90EE90}
\definecolor{wwwLightPink}{HTML}{FFB6C1}
\definecolor{wwwLightSalmon}{HTML}{FFA07A}
\definecolor{wwwLightSeaGreen}{HTML}{20B2AA}
\definecolor{wwwLightSkyBlue}{HTML}{87CEFA}
\definecolor{wwwLightSlateGray}{HTML}{778899}
\definecolor{wwwLightSteelBlue}{HTML}{B0C4DE}
\definecolor{wwwLightYellow}{HTML}{FFFFE0}
\definecolor{wwwLime}{HTML}{00FF00}
\definecolor{wwwLimeGreen}{HTML}{32CD32}
\definecolor{wwwLinen}{HTML}{FAF0E6}
\definecolor{wwwMagenta}{HTML}{FF00FF}
\definecolor{wwwMaroon}{HTML}{800000}
\definecolor{wwwMediumAquaMarine}{HTML}{66CDAA}
\definecolor{wwwMediumBlue}{HTML}{0000CD}
\definecolor{wwwMediumOrchid}{HTML}{BA55D3}
\definecolor{wwwMediumPurple}{HTML}{9370DB}
\definecolor{wwwMediumSeaGreen}{HTML}{3CB371}
\definecolor{wwwMediumSlateBlue}{HTML}{7B68EE}
\definecolor{wwwMediumSpringGreen}{HTML}{00FA9A}
\definecolor{wwwMediumTurquoise}{HTML}{48D1CC}
\definecolor{wwwMediumVioletRed}{HTML}{C71585}
\definecolor{wwwMidnightBlue}{HTML}{191970}
\definecolor{wwwMintCream}{HTML}{F5FFFA}
\definecolor{wwwMistyRose}{HTML}{FFE4E1}
\definecolor{wwwMoccasin}{HTML}{FFE4B5}
\definecolor{wwwNavajoWhite}{HTML}{FFDEAD}
\definecolor{wwwNavy}{HTML}{000080}
\definecolor{wwwOldLace}{HTML}{FDF5E6}
\definecolor{wwwOlive}{HTML}{808000}
\definecolor{wwwOliveDrab}{HTML}{6B8E23}
\definecolor{wwwOrange}{HTML}{FFA500}
\definecolor{wwwOrangeRed}{HTML}{FF4500}
\definecolor{wwwOrchid}{HTML}{DA70D6}
\definecolor{wwwPaleGoldenRod}{HTML}{EEE8AA}
\definecolor{wwwPaleGreen}{HTML}{98FB98}
\definecolor{wwwPaleTurquoise}{HTML}{AFEEEE}
\definecolor{wwwPaleVioletRed}{HTML}{DB7093}
\definecolor{wwwPapayaWhip}{HTML}{FFEFD5}
\definecolor{wwwPeachPuff}{HTML}{FFDAB9}
\definecolor{wwwPeru}{HTML}{CD853F}
\definecolor{wwwPink}{HTML}{FFC0CB}
\definecolor{wwwPlum}{HTML}{DDA0DD}
\definecolor{wwwPowderBlue}{HTML}{B0E0E6}
\definecolor{wwwPurple}{HTML}{800080}
\definecolor{wwwRebeccaPurple}{HTML}{663399}
\definecolor{wwwRed}{HTML}{FF0000}
\definecolor{wwwRosyBrown}{HTML}{BC8F8F}
\definecolor{wwwRoyalBlue}{HTML}{4169E1}
\definecolor{wwwSaddleBrown}{HTML}{8B4513}
\definecolor{wwwSalmon}{HTML}{FA8072}
\definecolor{wwwSandyBrown}{HTML}{F4A460}
\definecolor{wwwSeaGreen}{HTML}{2E8B57}
\definecolor{wwwSeaShell}{HTML}{FFF5EE}
\definecolor{wwwSienna}{HTML}{A0522D}
\definecolor{wwwSilver}{HTML}{C0C0C0}
\definecolor{wwwSkyBlue}{HTML}{87CEEB}
\definecolor{wwwSlateBlue}{HTML}{6A5ACD}
\definecolor{wwwSlateGray}{HTML}{708090}
\definecolor{wwwSnow}{HTML}{FFFAFA}
\definecolor{wwwSpringGreen}{HTML}{00FF7F}
\definecolor{wwwSteelBlue}{HTML}{4682B4}
\definecolor{wwwTan}{HTML}{D2B48C}
\definecolor{wwwTeal}{HTML}{008080}
\definecolor{wwwThistle}{HTML}{D8BFD8}
\definecolor{wwwTomato}{HTML}{FF6347}
\definecolor{wwwTurquoise}{HTML}{40E0D0}
\definecolor{wwwViolet}{HTML}{EE82EE}
\definecolor{wwwWheat}{HTML}{F5DEB3}
\definecolor{wwwWhite}{HTML}{FFFFFF}
\definecolor{wwwWhiteSmoke}{HTML}{F5F5F5}
\definecolor{wwwYellow}{HTML}{FFFF00}
\definecolor{wwwYellowGreen}{HTML}{9ACD32}


%----------------------------------------------------------------------------------------
% Standard colour for the whole book.
% NDunbar: It is possible to redefine ocre as some other colour, but it seems that there
% are places where text remains in the original Orange colour, rather than changing.
% For example, the TOC, the List of xxxx, Part pages where a mini-TOC is and page 
% numbers in the index. Everything else is fine.
%----------------------------------------------------------------------------------------
\definecolor{ocre}{RGB}{243,102,25} % Define the orange color used for highlighting throughout the book
%\definecolor{ocre}{named}{wwwDarkMagenta}



%----------------------------------------------------------------------------------------
% NDunbar: - I need program listings. Needs xcolor, used above.
% I also need multi row tables.
%----------------------------------------------------------------------------------------
\usepackage{listings}
\usepackage{multirow}
\usepackage{longtable}
\usepackage{datetime2}

%----------------------------------------------------------------------------------------
% NDunbar: - I need admonition boxes in colour.
%----------------------------------------------------------------------------------------
\definecolor{MyError}{named}{wwwRed}
\definecolor{myWarning}{named}{wwwOrange}
\definecolor{myNote}{named}{wwwBlue}
\definecolor{myInfo}{named}{wwwDarkGreen}


%----------------------------------------------------------------------------------------
% NDunbar: - Defaults for \begin{lstlisting} etc.
%----------------------------------------------------------------------------------------
\lstset{%
  backgroundcolor=\color{ocre!10},
  basicstyle=\small,
  breakatwhitespace=false,
  breaklines=false,
  captionpos=b,
  commentstyle=\color{wwwDarkGreen},
  deletekeywords={...},
  escapeinside={\%*}{*)},
  extendedchars=true,
  frame=leftline,
  framerule=4pt,
  keepspaces=true,
  keywordstyle=\color{blue},
  morekeywords={*,...},
  numbers=left,
  numbersep=10pt,
  numberstyle=\color{ocre},
  rulecolor=\color{ocre},
  showspaces=false,
  showstringspaces=false,
  showtabs=false,
  stepnumber=1,
  stringstyle=\color{wwwDarkOrchid},
  tabsize=2,
  title=\lstname,
  breaklines=true,  % Force code lines to break at the margin
  postbreak=\mbox{\textcolor{ocre}{\textbf{$\Longrightarrow$}}\space} % add a broken line indicator.
}


%----------------------------------------------------------------------------------------
%	FONTS
%----------------------------------------------------------------------------------------

\usepackage{avant} % Use the Avantgarde font for headings
%\usepackage{times} % Use the Times font for headings
\usepackage{mathptmx} % Use the Adobe Times Roman as the default text font together with math symbols from the Sym�bol, Chancery and Com�puter Modern fonts

\usepackage{microtype} % Slightly tweak font spacing for aesthetics
\usepackage[utf8]{inputenc} % Required for including letters with accents
\usepackage[T1]{fontenc} % Use 8-bit encoding that has 256 glyphs

%----------------------------------------------------------------------------------------
%	BIBLIOGRAPHY AND INDEX
%----------------------------------------------------------------------------------------

\usepackage[style=alphabetic,citestyle=numeric,sorting=nyt,sortcites=true,autopunct=true,babel=hyphen,hyperref=true,abbreviate=false,backref=true,backend=biber]{biblatex}
\addbibresource{bibliography.bib} % BibTeX bibliography file
\defbibheading{bibempty}{}

\usepackage{calc} % For simpler calculation - used for spacing the index letter headings correctly
\usepackage{makeidx} % Required to make an index
\makeindex % Tells LaTeX to create the files required for indexing

%----------------------------------------------------------------------------------------
%	MAIN TABLE OF CONTENTS
%----------------------------------------------------------------------------------------

\usepackage{titletoc} % Required for manipulating the table of contents

\contentsmargin{0cm} % Removes the default margin

% Part text styling
\titlecontents{part}[0cm]
{\addvspace{20pt}\centering\large\bfseries}
{}
{}
{}

% Chapter text styling in the main TOC.
\titlecontents{chapter}[1.25cm] % Indentation
{\addvspace{12pt}\large\sffamily\bfseries} % Spacing and font options for chapters
% The Chapter Number at the start of the line.
%{\color{ocre!60}\contentslabel[\Large\thecontentslabel]{1.25cm}\color{ocre}} 
{\color{ocre!60}\contentslabel[\Large\thecontentslabel]{1.25cm}} % NDunbar
%{\color{ocre}}  
{} % NDunbar
% The dots and the page number at the end of the line.
{\color{ocre!60}\normalsize\;\titlerule*[.5pc]{.}\;\thecontentspage}

% Section text styling in the main TOC.
\titlecontents{section}[1.25cm] % Indentation
{\addvspace{3pt}\sffamily\bfseries} % Spacing and font options for sections
% The section number at the start of the line.
%{\contentslabel[\color{black}\thecontentslabel]{1.25cm}} % Section number
{\contentslabel[\color{black}\thecontentslabel]{1.25cm}} % NDunbar
{}
{\hfill\color{black}\thecontentspage} % Page number
[]

% Subsection text styling
\titlecontents{subsection}[1.25cm] % Indentation
{\addvspace{1pt}\sffamily\small} % Spacing and font options for subsections
{\contentslabel[\thecontentslabel]{1.25cm}} % Subsection number
{}
{\ \titlerule*[.5pc]{.}\;\thecontentspage} % Page number
[]

% List of figures
\titlecontents{figure}[0em]
{\addvspace{-5pt}\sffamily}  % NDunbar
%{\addvspace{1pt}}
{\thecontentslabel\hspace*{1em}}
{}
{\ \titlerule*[.5pc]{.}\;\thecontentspage}
[]

% List of tables
\titlecontents{table}[0em]
{\addvspace{-5pt}\sffamily} % NDunbar
%{\addvspace{1pt}}
{\thecontentslabel\hspace*{1em}}
{}
{\ \titlerule*[.5pc]{.}\;\thecontentspage}
[]

%----------------------------------------------------------------------------------------
%	MINI TABLE OF CONTENTS IN PART HEADS
%----------------------------------------------------------------------------------------

% Chapter text styling
\titlecontents{lchapter}[0em] % Indenting
{\addvspace{15pt}\large\sffamily\bfseries} % Spacing and font options for chapters
{\color{ocre}\contentslabel[\Large\thecontentslabel]{1.25cm}\color{ocre}} % Chapter number
{}  
{\color{ocre}\normalsize\sffamily\bfseries\;\titlerule*[.5pc]{.}\;\thecontentspage} % Page number

% Section text styling
\titlecontents{lsection}[0em] % Indenting
{\sffamily\small} % Spacing and font options for sections
{\contentslabel[\thecontentslabel]{1.25cm}} % Section number
{}
{}

% Subsection text styling
\titlecontents{lsubsection}[.5em] % Indentation
{\normalfont\footnotesize\sffamily} % Font settings
{}
{}
{}

%----------------------------------------------------------------------------------------
%	PAGE HEADERS
%----------------------------------------------------------------------------------------

\usepackage{fancyhdr} % Required for header and footer configuration

\pagestyle{fancy}
\renewcommand{\chaptermark}[1]{\markboth{\sffamily\normalsize\bfseries\chaptername\ \thechapter.\ #1}{}} % Chapter text font settings
\renewcommand{\sectionmark}[1]{\markright{\sffamily\normalsize\thesection\hspace{5pt}#1}{}} % Section text font settings
\fancyhf{} \fancyhead[LE,RO]{\sffamily\normalsize\thepage} % Font setting for the page number in the header
\fancyhead[LO]{\rightmark} % Print the nearest section name on the left side of odd pages
\fancyhead[RE]{\leftmark} % Print the current chapter name on the right side of even pages
\renewcommand{\headrulewidth}{0.5pt} % Width of the rule under the header
\addtolength{\headheight}{2.5pt} % Increase the spacing around the header slightly
\renewcommand{\footrulewidth}{0pt} % Removes the rule in the footer
\fancypagestyle{plain}{\fancyhead{}\renewcommand{\headrulewidth}{0pt}} % Style for when a plain pagestyle is specified

% Removes the header from odd empty pages at the end of chapters
\makeatletter
\renewcommand{\cleardoublepage}{
\clearpage\ifodd\c@page\else
\hbox{}
\vspace*{\fill}
\thispagestyle{empty}
\newpage
\fi}

%----------------------------------------------------------------------------------------
%	THEOREM STYLES
%----------------------------------------------------------------------------------------

\usepackage{amsmath,amsfonts,amssymb,amsthm} % For math equations, theorems, symbols, etc

\newcommand{\intoo}[2]{\mathopen{]}#1\,;#2\mathclose{[}}
\newcommand{\ud}{\mathop{\mathrm{{}d}}\mathopen{}}
\newcommand{\intff}[2]{\mathopen{[}#1\,;#2\mathclose{]}}
\newtheorem{notation}{Notation}[chapter]

% Boxed/framed environments
\newtheoremstyle{ocrenumbox}% % Theorem style name
{0pt}% Space above
{0pt}% Space below
{\normalfont}% % Body font
{}% Indent amount
{\small\bf\sffamily\color{ocre}}% % Theorem head font
{\;}% Punctuation after theorem head
{0.25em}% Space after theorem head
{\small\sffamily\color{ocre}\thmname{#1}\nobreakspace\thmnumber{\@ifnotempty{#1}{}\@upn{#2}}% Theorem text (e.g. Theorem 2.1)
\thmnote{\nobreakspace\the\thm@notefont\sffamily\bfseries\color{black}---\nobreakspace#3.}} % Optional theorem note
\renewcommand{\qedsymbol}{$\blacksquare$}% Optional qed square

\newtheoremstyle{blacknumex}% Theorem style name
{5pt}% Space above
{5pt}% Space below
{\normalfont}% Body font
{} % Indent amount
{\small\bf\sffamily}% Theorem head font
{\;}% Punctuation after theorem head
{0.25em}% Space after theorem head
{\small\sffamily{\tiny\ensuremath{\blacksquare}}\nobreakspace\thmname{#1}\nobreakspace\thmnumber{\@ifnotempty{#1}{}\@upn{#2}}% Theorem text (e.g. Theorem 2.1)
\thmnote{\nobreakspace\the\thm@notefont\sffamily\bfseries---\nobreakspace#3.}}% Optional theorem note

\newtheoremstyle{blacknumbox} % Theorem style name
{0pt}% Space above
{0pt}% Space below
{\normalfont}% Body font
{}% Indent amount
{\small\bf\sffamily}% Theorem head font
{\;}% Punctuation after theorem head
{0.25em}% Space after theorem head
{\small\sffamily\thmname{#1}\nobreakspace\thmnumber{\@ifnotempty{#1}{}\@upn{#2}}% Theorem text (e.g. Theorem 2.1)
\thmnote{\nobreakspace\the\thm@notefont\sffamily\bfseries---\nobreakspace#3.}}% Optional theorem note

% Non-boxed/non-framed environments
\newtheoremstyle{ocrenum}% % Theorem style name
{5pt}% Space above
{5pt}% Space below
{\normalfont}% % Body font
{}% Indent amount
{\small\bf\sffamily\color{ocre}}% % Theorem head font
{\;}% Punctuation after theorem head
{0.25em}% Space after theorem head
{\small\sffamily\color{ocre}\thmname{#1}\nobreakspace\thmnumber{\@ifnotempty{#1}{}\@upn{#2}}% Theorem text (e.g. Theorem 2.1)
\thmnote{\nobreakspace\the\thm@notefont\sffamily\bfseries\color{black}---\nobreakspace#3.}} % Optional theorem note
\renewcommand{\qedsymbol}{$\blacksquare$}% Optional qed square
\makeatother

% Defines the theorem text style for each type of theorem to one of the three styles above
\newcounter{dummy} 
\numberwithin{dummy}{section}
\theoremstyle{ocrenumbox}
\newtheorem{theoremeT}[dummy]{Theorem}
\newtheorem{problem}{Problem}[chapter]
\newtheorem{exerciseT}{Exercise}[chapter]
\theoremstyle{blacknumex}
\newtheorem{exampleT}{Example}[chapter]
\theoremstyle{blacknumbox}
\newtheorem{vocabulary}{Vocabulary}[chapter]
\newtheorem{definitionT}{Definition}[section]
\newtheorem{corollaryT}[dummy]{Corollary}
\theoremstyle{ocrenum}
\newtheorem{proposition}[dummy]{Proposition}

%----------------------------------------------------------------------------------------
%	DEFINITION OF COLORED BOXES
%----------------------------------------------------------------------------------------

\RequirePackage[framemethod=default]{mdframed} % Required for creating the theorem, definition, exercise and corollary boxes

% Theorem box
\newmdenv[skipabove=7pt,
skipbelow=7pt,
backgroundcolor=black!5,
linecolor=ocre,
innerleftmargin=5pt,
innerrightmargin=5pt,
innertopmargin=5pt,
leftmargin=0cm,
rightmargin=0cm,
innerbottommargin=5pt]{tBox}

% Exercise box	  
\newmdenv[skipabove=7pt,
skipbelow=7pt,
rightline=false,
leftline=true,
topline=false,
bottomline=false,
backgroundcolor=ocre!10,
linecolor=ocre,
innerleftmargin=5pt,
innerrightmargin=5pt,
innertopmargin=5pt,
innerbottommargin=5pt,
leftmargin=0cm,
rightmargin=0cm,
linewidth=4pt]{eBox}	

% Definition box
\newmdenv[skipabove=7pt,
skipbelow=7pt,
rightline=false,
leftline=true,
topline=false,
bottomline=false,
linecolor=ocre,
innerleftmargin=5pt,
innerrightmargin=5pt,
innertopmargin=0pt,
leftmargin=0cm,
rightmargin=0cm,
linewidth=4pt,
innerbottommargin=0pt]{dBox}	

% Corollary box
\newmdenv[skipabove=7pt,
skipbelow=7pt,
rightline=false,
leftline=true,
topline=false,
bottomline=false,
linecolor=gray,
backgroundcolor=black!5,
innerleftmargin=5pt,
innerrightmargin=5pt,
innertopmargin=5pt,
leftmargin=0cm,
rightmargin=0cm,
linewidth=4pt,
innerbottommargin=5pt]{cBox}

% Creates an environment for each type of theorem and assigns it a theorem text style from the "Theorem Styles" section above and a colored box from above
\newenvironment{theorem}{\begin{tBox}\begin{theoremeT}}{\end{theoremeT}\end{tBox}}
\newenvironment{exercise}{\begin{eBox}\begin{exerciseT}}{\hfill{\color{ocre}\tiny\ensuremath{\blacksquare}}\end{exerciseT}\end{eBox}}				  
\newenvironment{definition}{\begin{dBox}\begin{definitionT}}{\end{definitionT}\end{dBox}}	
\newenvironment{example}{\begin{exampleT}}{\hfill{\tiny\ensuremath{\blacksquare}}\end{exampleT}}		
\newenvironment{corollary}{\begin{cBox}\begin{corollaryT}}{\end{corollaryT}\end{cBox}}	

%----------------------------------------------------------------------------------------
%	REMARK ENVIRONMENT
%----------------------------------------------------------------------------------------

\newenvironment{remark}{\par\vspace{10pt}\small % Vertical white space above the remark and smaller font size
\begin{list}{}{
\leftmargin=35pt % Indentation on the left
\rightmargin=25pt}\item\ignorespaces % Indentation on the right
\makebox[-2.5pt]{\begin{tikzpicture}[overlay]
\node[draw=ocre!60,line width=1pt,circle,fill=ocre!25,font=\sffamily\bfseries,inner sep=2pt,outer sep=0pt] at (-15pt,-0pt){\textcolor{ocre}{R}};\end{tikzpicture}} % Orange R in a circle
\advance\baselineskip -1pt}{\end{list}\vskip5pt} % Tighter line spacing and white space after remark

%----------------------------------------------------------------------------------------
%	NDunbar: WARNING ENVIRONMENT
%----------------------------------------------------------------------------------------

\newenvironment{warning}{\par\vspace{10pt}\small % Vertical white space above the remark and smaller font size
\begin{list}{}{
\leftmargin=35pt % Indentation on the left
\rightmargin=25pt}\item\ignorespaces % Indentation on the right
\makebox[-2.5pt]{\begin{tikzpicture}[overlay]
\node[draw=myWarning!60,line width=1pt,circle,fill=myWarning!25,font=\sffamily\bfseries,inner sep=2pt,outer sep=0pt] at (-25pt,-10pt){\textcolor{myWarning}{Warn}};\end{tikzpicture}} % Amber text in a circle
\advance\baselineskip -1pt}{\end{list}\vskip5pt} % Tighter line spacing and white space after Warning

%----------------------------------------------------------------------------------------
%	NDunbar: NOTE ENVIRONMENT
%----------------------------------------------------------------------------------------

\newenvironment{note}{\par\vspace{10pt}\small % Vertical white space above the remark and smaller font size
\begin{list}{}{
\leftmargin=35pt % Indentation on the left
\rightmargin=25pt}\item\ignorespaces % Indentation on the right
\makebox[-2.5pt]{\begin{tikzpicture}[overlay]
\node[draw=myNote!60,line width=1pt,circle,fill=myNote!25,font=\sffamily\bfseries,inner sep=2pt,outer sep=0pt] at (-25pt,-10pt){\textcolor{myNote}{Note}};\end{tikzpicture}} % Blue text in a circle
\advance\baselineskip -1pt}{\end{list}\vskip5pt} % Tighter line spacing and white space after Warning

%----------------------------------------------------------------------------------------
%	NDunbar: INFO ENVIRONMENT
%----------------------------------------------------------------------------------------

\newenvironment{info}{\par\vspace{10pt}\small % Vertical white space above the remark and smaller font size
\begin{list}{}{
\leftmargin=35pt % Indentation on the left
\rightmargin=25pt}\item\ignorespaces % Indentation on the right
\makebox[-2.5pt]{\begin{tikzpicture}[overlay]
\node[draw=myInfo!60,line width=1pt,circle,fill=myInfo!25,font=\sffamily\bfseries,inner sep=2pt,outer sep=0pt] at (-25pt,-10pt){\textcolor{myInfo}{Info}};\end{tikzpicture}} % Green text in a circle
\advance\baselineskip -1pt}{\end{list}\vskip5pt} % Tighter line spacing and white space after Warning

%----------------------------------------------------------------------------------------
%	NDunbar: ERROR ENVIRONMENT
%----------------------------------------------------------------------------------------

\newenvironment{error}{\par\vspace{10pt}\small % Vertical white space above the remark and smaller font size
\begin{list}{}{
\leftmargin=35pt % Indentation on the left
\rightmargin=25pt}\item\ignorespaces % Indentation on the right
\makebox[-2.5pt]{\begin{tikzpicture}[overlay]
\node[draw=MyError!60,line width=1pt,circle,fill=MyError!25,font=\sffamily\bfseries,inner sep=2pt,outer sep=0pt] at (-25pt,-10pt){\textcolor{MyError}{Error}};\end{tikzpicture}} % Red text in a circle
\advance\baselineskip -1pt}{\end{list}\vskip5pt} % Tighter line spacing and white space after Warning

%----------------------------------------------------------------------------------------
%	SECTION NUMBERING IN THE MARGIN
%----------------------------------------------------------------------------------------

\makeatletter
\renewcommand{\@seccntformat}[1]{\llap{\textcolor{ocre}{\csname the#1\endcsname}\hspace{1em}}}                    
\renewcommand{\section}{\@startsection{section}{1}{\z@}
{-4ex \@plus -1ex \@minus -.4ex}
{1ex \@plus.2ex }
{\normalfont\large\sffamily\bfseries}}
\renewcommand{\subsection}{\@startsection {subsection}{2}{\z@}
{-3ex \@plus -0.1ex \@minus -.4ex}
{0.5ex \@plus.2ex }
{\normalfont\sffamily\bfseries}}
\renewcommand{\subsubsection}{\@startsection {subsubsection}{3}{\z@}
{-2ex \@plus -0.1ex \@minus -.2ex}
{.2ex \@plus.2ex }
{\normalfont\small\sffamily\bfseries}}                        
\renewcommand\paragraph{\@startsection{paragraph}{4}{\z@}
{-2ex \@plus-.2ex \@minus .2ex}
{.1ex}
{\normalfont\small\sffamily\bfseries}}

%----------------------------------------------------------------------------------------
%	PART HEADINGS
%----------------------------------------------------------------------------------------

% numbered part in the table of contents
\newcommand{\@mypartnumtocformat}[2]{%
\setlength\fboxsep{0pt}%
\noindent\colorbox{ocre!20}{\strut\parbox[c][.7cm]{\ecart}{\color{ocre!70}\Large\sffamily\bfseries\centering#1}}\hskip\esp\colorbox{ocre!40}{\strut\parbox[c][.7cm]{\linewidth-\ecart-\esp}{\Large\sffamily\centering#2}}}%
%%%%%%%%%%%%%%%%%%%%%%%%%%%%%%%%%%
% unnumbered part in the table of contents
\newcommand{\@myparttocformat}[1]{%
\setlength\fboxsep{0pt}%
\noindent\colorbox{ocre!40}{\strut\parbox[c][.7cm]{\linewidth}{\Large\sffamily\centering#1}}}%
%%%%%%%%%%%%%%%%%%%%%%%%%%%%%%%%%%
\newlength\esp
\setlength\esp{4pt}
\newlength\ecart
\setlength\ecart{1.2cm-\esp}
\newcommand{\thepartimage}{}%
\newcommand{\partimage}[1]{\renewcommand{\thepartimage}{#1}}%
\def\@part[#1]#2{%
\ifnum \c@secnumdepth >-2\relax%
\refstepcounter{part}%
\addcontentsline{toc}{part}{\texorpdfstring{\protect\@mypartnumtocformat{\thepart}{#1}}{\partname~\thepart\ ---\ #1}}
\else%
\addcontentsline{toc}{part}{\texorpdfstring{\protect\@myparttocformat{#1}}{#1}}%
\fi%
\startcontents%
\markboth{}{}%
{\thispagestyle{empty}%
\begin{tikzpicture}[remember picture,overlay]%
\node at (current page.north west){\begin{tikzpicture}[remember picture,overlay]%	
\fill[ocre!20](0cm,0cm) rectangle (\paperwidth,-\paperheight);
\node[anchor=north] at (4cm,-3.25cm){\color{ocre!40}\fontsize{220}{100}\sffamily\bfseries\@Roman\c@part}; 
\node[anchor=south east] at (\paperwidth-1cm,-\paperheight+1cm){\parbox[t][][t]{8.5cm}{
\printcontents{l}{0}{\setcounter{tocdepth}{1}}%
}};
\node[anchor=north east] at (\paperwidth-1.5cm,-3.25cm){\parbox[t][][t]{15cm}{\strut\raggedleft\color{white}\fontsize{30}{30}\sffamily\bfseries#2}};
\end{tikzpicture}};
\end{tikzpicture}}%
\@endpart}
\def\@spart#1{%
\startcontents%
\phantomsection
{\thispagestyle{empty}%
\begin{tikzpicture}[remember picture,overlay]%
\node at (current page.north west){\begin{tikzpicture}[remember picture,overlay]%	
\fill[ocre!20](0cm,0cm) rectangle (\paperwidth,-\paperheight);
\node[anchor=north east] at (\paperwidth-1.5cm,-3.25cm){\parbox[t][][t]{15cm}{\strut\raggedleft\color{white}\fontsize{30}{30}\sffamily\bfseries#1}};
\end{tikzpicture}};
\end{tikzpicture}}
\addcontentsline{toc}{part}{\texorpdfstring{%
\setlength\fboxsep{0pt}%
\noindent\protect\colorbox{ocre!40}{\strut\protect\parbox[c][.7cm]{\linewidth}{\Large\sffamily\protect\centering #1\quad\mbox{}}}}{#1}}%
\@endpart}
\def\@endpart{\vfil\newpage
\if@twoside
\if@openright
\null
\thispagestyle{empty}%
\newpage
\fi
\fi
\if@tempswa
\twocolumn
\fi}

%----------------------------------------------------------------------------------------
%	CHAPTER HEADINGS
%----------------------------------------------------------------------------------------

\newcommand{\thechapterimage}{}%
\newcommand{\chapterimage}[1]{\renewcommand{\thechapterimage}{#1}}%
\def\@makechapterhead#1{%
{\parindent \z@ \raggedright \normalfont
\ifnum \c@secnumdepth >\m@ne
%----------------------------------------------------------------------------------------
% NDunbar: This bit does the main matter ...
%----------------------------------------------------------------------------------------
\if@mainmatter
\begin{tikzpicture}[remember picture,overlay]
\node at (current page.north west)
{\begin{tikzpicture}[remember picture,overlay]
%\node[anchor=north west,inner sep=0pt] at (0,0) {\includegraphics[width=\paperwidth]{\thechapterimage}};
\node[anchor=north west,inner sep=0pt, opacity=0.5] at (0,0) {\includegraphics[width=\paperwidth]{\thechapterimage}}; % NDunbar: Added opacity.
\draw[anchor=west] (\Gm@lmargin,-9cm) node [line width=2pt,rounded corners=15pt,draw=ocre,fill=white,fill opacity=0.5,inner sep=15pt]{\strut\makebox[22cm]{}};
\draw[anchor=west] (\Gm@lmargin+.3cm,-9cm) node {\huge\sffamily\bfseries\color{black}\thechapter. #1\strut}; 
\end{tikzpicture}};
\end{tikzpicture}
\else
%----------------------------------------------------------------------------------------
% NDunbar: And this does frontmatter and backmatter. Etc.
%----------------------------------------------------------------------------------------
\begin{tikzpicture}[remember picture,overlay]
\node at (current page.north west)
{\begin{tikzpicture}[remember picture,overlay]
%\node[anchor=north west,inner sep=0pt] at (0,0) {\includegraphics[width=\paperwidth]{\thechapterimage}};
\node[anchor=north west,inner sep=0pt, opacity=0.5] at (0,0) {\includegraphics[width=\paperwidth]{\thechapterimage}}; NDunbar: Added opacity.
\draw[anchor=west] (\Gm@lmargin,-9cm) node [line width=2pt,rounded corners=15pt,draw=ocre,fill=white,fill opacity=0.5,inner sep=15pt]{\strut\makebox[22cm]{}};
\draw[anchor=west] (\Gm@lmargin+.3cm,-9cm) node {\huge\sffamily\bfseries\color{black}#1\strut};
\end{tikzpicture}};
\end{tikzpicture}
\fi\fi\par\vspace*{270\p@}}}

%-------------------------------------------

\def\@makeschapterhead#1{%
\begin{tikzpicture}[remember picture,overlay]
\node at (current page.north west)
{\begin{tikzpicture}[remember picture,overlay]
\node[anchor=north west,inner sep=0pt, opacity=0.5] at (0,0) {\includegraphics[width=\paperwidth]{\thechapterimage}}; % NDunbar: Added opacity for List Of xxx pages..
\draw[anchor=west] (\Gm@lmargin,-9cm) node [line width=2pt,rounded corners=15pt,draw=ocre,fill=white,fill opacity=0.5,inner sep=15pt]{\strut\makebox[22cm]{}};
\draw[anchor=west] (\Gm@lmargin+.3cm,-9cm) node {\huge\sffamily\bfseries\color{black}#1\strut};
\end{tikzpicture}};
\end{tikzpicture}
\par\vspace*{270\p@}}
\makeatother

%----------------------------------------------------------------------------------------
%	HYPERLINKS IN THE DOCUMENTS
% NDunbar: Enabled colorlinks from false to true.
%----------------------------------------------------------------------------------------

\usepackage{hyperref}
% NDUNBAR: Original line has warnings.
% \hypersetup{hidelinks,backref=true,pagebackref=true,hyperindex=true,colorlinks=true,breaklinks=true,urlcolor=ocre,bookmarks=true,bookmarksopen=false}

\hypersetup{hidelinks,colorlinks=true,breaklinks=true,urlcolor=ocre,bookmarksopen=false}

\usepackage{bookmark}
\bookmarksetup{
open,
numbered,
addtohook={%
\ifnum\bookmarkget{level}=0 % chapter
\bookmarksetup{bold}%
\fi
\ifnum\bookmarkget{level}=-1 % part
\bookmarksetup{color=ocre,bold}%
\fi
}
}


%----------------------------------------------------------------------------------------
% NDunbar: Paragraph indendation and spacing. Needed due to lots of small one line
% paragraphs and listings etc.
%----------------------------------------------------------------------------------------
\setlength{\parskip}{6pt}   % For a 6pt gap between paragraphs.
\setlength{\parindent}{0pt} % For a lack of paragraph indenting
