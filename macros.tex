%----------------------------------------------------------------------------------------
% NDunbar: A few useful (?) macros to save typing.
%----------------------------------------------------------------------------------------

%----------------------------------------------------------------------------------------
% These need to be used with the text *twice*, Once to add it to the main document "as is" 
% and once again to add it to the main index, under the sub-index named. For example:
%
% The C and C++ compiler used for AVR programming is is avr-gcc\program{avr-gcc} ...
%----------------------------------------------------------------------------------------
\DeclareRobustCommand{\program}[1]{\index{Programs and applications!#1}}
\DeclareRobustCommand{\mc6800x}[1]{\index{MC6800x Instructions!#1}}
\DeclareRobustCommand{\address}[1]{\index{Addressing Modes!#1}}

%----------------------------------------------------------------------------------------
% These generate the text in the main body in teletype mode as well as 
% adding them to the index:
% Use \vector(BP\_INIT} to link new functions to SuperBasic.
%----------------------------------------------------------------------------------------
\DeclareRobustCommand{\vector}[1]{\texttt{#1}\index{Vectored Utilities!#1}}
\DeclareRobustCommand{\trap}[1]{\texttt{#1}\index{Trap Calls!#1}}
\DeclareRobustCommand{\pe}[1]{\texttt{#1}\index{Pointer Environment Vectors!#1}}

\DeclareRobustCommand{\listing}[1]{#1\index{Code Listings!#1}}

%----------------------------------------------------------------------------------------
% Index a word and add it to the text exactly as it is.
%----------------------------------------------------------------------------------------
\DeclareRobustCommand{\indexasis}[1]{#1\index{#1}}

%----------------------------------------------------------------------------------------
% These generate text in a certain style, see the actual command for details,
% when used. These do not get indexed.
%----------------------------------------------------------------------------------------
\DeclareRobustCommand{\opcode}[1]{\texttt{#1}}
\DeclareRobustCommand{\inline}[1]{\texttt{#1}}



%----------------------------------------------------------------------------------------
% When using PanDoc to convert from RST to LaTeX,it generated a \tightlist for bullet
%  lists. Unfortunately, that's not known (at least in MikTex) so this helps.
%----------------------------------------------------------------------------------------
\providecommand{\tightlist}{%
  \setlength{\itemsep}{0pt}\setlength{\parskip}{0pt}}