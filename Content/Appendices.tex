\begin{appendix}

\chapter{How the AVRBook Evolved}
\label{appendix-a-how-it-was-done}%\hyperlabel{appendix-a-how-it-was-done}%

\section*{Why this Book?}\label{why-this-book}

This book originally started as a manner for me to collect useful information, for my own use, in one easy to find place\footnote{I have the memory capacity of a newt, so I need to have things written down!}.

There are many places on the internet, on You Tube, in various books which I have bought (or had bought for me) and so on. Some (\emph{most}) are a lot better than this one - but this is mine so I'm not all that bothered, I can update this book as I need to.

Anyway, the short story is, I needed to upgrade from plain Arduino, to AVR, and I collected a lot of stuff from the above sources, and needed to have it all together in one place.

That's why.

\section*{Creating the Book}\label{creating-the-book}

The book was created in plain text files, originally in \href{https://en.wikipedia.org/wiki/ReStructuredText}{ReStructuredText} mode, written (and edited) on both Windows\footnote{At work, in my lunch hour}, or on Linux - which I use for all my personal and business needs at home.

These RST files were simply a quick way to gather notes. They were then later enhanced by adding more detail and/or example code, and converted to \LaTeX{} by judicious use of the \href{https://pandoc.org/}{Pandoc} text conversion utility, which I \emph{highly} recommend.

The converted \LaTeX{} files were then further enhanced, indexed, tidied up, etc using \href{http://www.texstudio.org/}{TexStudio} which runs on both Windows and Linux, so I had the same development environment in both locations. Handy. 

The book itself, was created using a \LaTeX{} template called [the] \href{https://www.latextemplates.com/template/the-legrand-orange-book}{Legrand Orange Book Template} created by Mathias Legrand. It is thanks to him that you get this book for free, because the licence terms of the book template specify no commercial use. I'm happy with that myself.

The front cover image on this book is taken from the book \emph{Kunstformen der Natur} by German biologist Ernst Haeckel. The book was published between 1899 and 1904. The image used is of various \emph{Polycystines} which are a specific kind of micro-fossil.

You can read about them on \href{https://en.wikipedia.org/wiki/Polycystine}{Wikipedia} and there is a brief overview of the above book, also on \href{https://en.wikipedia.org/wiki/Kunstformen_der_Natur}{Wikipedia},
which shows a number of other images taken from the book. (Some of which I considered before choosing the current one!)

Polycystines have absolutely nothing to do with the Arduino or AVR microcontrollers - in fact, I suspect they died out before electronics were invented - but I liked the image, and decided that it would make a good cover for the book and a decent enough chapter heading image too.





\end{appendix}
