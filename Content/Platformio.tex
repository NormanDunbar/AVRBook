\chapter{Installing PlatformIO}\label{installing-platformio}

\section{PlatformIO}\label{platformio}\index{PlatformIO}

\href{http://platformio.org/}{PlatformIO} is a command line\footnote{Panic
 not Windows users, the command line is not your mortal enemy. In fact,
 you can get stuff done a lot quicker in the command line, \emph{most}
 of the time. Once you get to know it of course.} and IDE based utility
that allows you to write plain C or C++ code to upload to your Arduino,
or, almost any other embedded microcontroller that is known to man,
woman or hamster! At the time of writing, 5th July 2017, PlatformIO
supported:

\begin{itemize}
\item
  23 Development Platforms
\item
  13 Frameworks
\item
  412 Embedded Boards
\item
  61 Project Examples
\item
  1,738 Libraries
\item
  8,084 Library Examples
\end{itemize}

You can use it to replace the Arduino IDE too, if you wish, as it
understands plain Arduino code as well as AVR C code. Arduino sketches
can be imported unchanged, worked on, compiled and uploaded without any
difficulty.

PlatformIO comes in two main flavours:

\begin{itemize}
\item
  The command line version\footnote{See previous footnote!}
\item
  An IDE version for the Atom editor, or Visual Studio. This also allows
  you to run the command line commands without having to install that
  package separately.
\end{itemize}

In addition, there's a lot of documentation on the web site about how
you can install the PlatformIO tools into a large number of different
IDEs, some of which you may already be using.

\subsection{Installing Atom}\label{installing-atom}

As a lot of Windows people \emph{might} not be fully acquainted with the command
line, I've decided to base the remainder of this book on using the Atom editor
from Github, and the \emph{PlatformIO IDE} package for it. 

Installing is simple. Go to \href{https://atom.io}{https://atom.io} where you will be presented with a download link for your operating system. If that is suitable, click it, otherwise choose the `Other Platforms' link and pick the best option from the list. It's best to be wary of the pre-release versions, stick with the current versions instead.

Windows users will get an installer. Linux users will get a package, rpm, deb for example, which can be installed with the default package manager.

That's pretty much all there is to it.

\subsection{Installing PlatformIO in
Atom}\label{installing-platformio-in-atom}

We will install the following 2 PlatformIO packages for Atom:

\begin{itemize}
\item
  platformio-ide (version 2.0.0 beta 7 is the latest at the time of
  writing)
\item
  platformio-ide-debugger (version 1.2.3 is the latest at the time of
  writing)
\end{itemize}

Open Atom and select the \inline{edit -> Preferences} menu option.
When the \inline{Settings} tab appears, look down to find the option labelled
\inline{+ Install} and choose it.

Search for \inline{platformio} and a list of available options will soon be
displayed. Find the two packages mentioned above and click the \inline{install}
button. Wait \ldots{}.

Eventually you will be prompted to restart Atom, so do so.

If, on the restart, you are prompted to Install Clang go ahead and do
so, if you wish, or select \inline{Disable Code Completion}. Clang
will have to be installed using your package manager - it's not a
package for Atom but for the entire system. Full details are
\href{http://docs.platformio.org/en/latest/ide/atom.html\#ii-clang-for-intelligent-code-completion}{here}.

Full instructions on installing Atom and PlatformIO packages can be
found
\href{http://docs.platformio.org/en/latest/ide/atom.html\#clang-for-intelligent-code-completion}{here}.

You will know that you have installed PlatformIO when you restart Atom
and see a `bug face' looking back at you from the PlatformIO welcome
screen.

\section{PlatformIO Quick Start}\label{platformio-quick-start}

\subsection{Create the Blink Project}\label{create-the-blink-project}

Open the Atom editor, if not already running, and make sure that you can
see the PlatformIO Welcome Screen. Click on \inline{+ New Project} to start the
New Project Wizard.

You must first select a board. As I'm using an Arduino Nano, for this
example, that's what I'll be selecting. You may choose something
different.

You will hopefully notice that when you have selected a board, you get
to select another, and another \ldots{}. this is a feature of PlatformIO
in that it allows you to use the same source code for multiple boards
and/or microcontrollers. This is very nice as I sometimes build for the Nano, the Uno,
the Duemilenova and for the stand-alone AtTiny85 - so I can pretty much use the same code
for all of the above. (The AtTiny85 needs a few \inline{\#define}s to compile the same code though!)

For now, however, let's stick with a single board.

Now select a directory to save the code in. My choice is:

 \inline{/home/norman/SourceCode/PlatformIO/AVRBlink}

you might want to choose something different, especially if you are on
Windows!

\begin{note}
You can obtain the source code, but not the PlatformIO project files etc, for all the examples in this book from \href{https://github.com/NormanDunbar/AVRBook}{GitHub} - look for the folder named \inline{SourceCode}. The code will be found in the same part and/or chapter as the book.
\end{note}

Click the `Process' button.

The first time PlatformIO is used, it needs to connect to the internet
to install the appropriate tools for the board you have chosen. In the
case of the Nano, this is \inline{platform: atmelavr}. Just wait while it does what it needs to do.

When the processing has completed, the output directory that you chose
above will be populated by a number of files and directories:

\begin{itemize}
\item
  A directory named \inline{lib} and a file, \inline{readme.txt} within.
  The README file has all the information you will need if you intend to
  write some libraries for this particular project. The source code for
  the libraries should be kept in this directory - with a separate directory for each separate library.
\item
  A directory named \inline{src}. This is where you will create your own
  project's source code.
\item
  A file named \inline{.gitignore}. This (hidden on Linux) file tells git, if in use for
  version control, to ignore a number of files that are not required in
  normal circumstances, usually files that can be recreated at will.
\item
  A file named \inline{.travis.yml}. This (hidden on Linux) file is a YAML source file that
  tells the Travis `Continuous Integration' system what to do whenever
  something in the project changes. We will not be using this feature.
\item
  A file named \inline{platformio.ini}. This file tells PlatformIO how the
  project will be built and uploaded to the microcontroller, for each
  board you selected when creating the project.
\end{itemize}

\subsection{Write the Blink Code}\label{write-the-blink-code}

Now we have our project (AVRBlink) created, we need to write some code, so:

\begin{itemize}
\item
  Right-click the \inline{src} folder in the \inline{Project} tab, and select
  \inline{New File} from the context menu that appears.
\item
  Enter the file name when prompted, I chose \inline{AVRBlink.c}, but you can name
  the file as you wish.
\end{itemize}

The new file opens in the editor, so add the following code:

\begin{lstlisting}[language=C,caption={AVRBlink.c}]
// This must also be done before anything else.
#ifndef F_CPU
#define F_CPU 16000000UL // 16 MHz clock speed
#endif

#include <avr/io.h>
#include <util/delay.h>

int main(void)
{
  // SETUP goes here.
  // Makes D13 aka PB5 (PORTB, Pin5) an output pin
  DDRB = (1 << PB5); 

  // And here is what Arduino calls loop.
  while(1) //infinite loop
  {
    // Turn on D13 LED.
    PORTB =  (1 << PB5);
    _delay_ms(1000);

    // Turn off D13 LED.
    PORTB= 0x00; //Turns OFF All LEDs
    _delay_ms(1000); //1 second delay
  }
}
\end{lstlisting}

\subsection{Compile the Code}\label{compile-the-code}

Save the file, \inline{File -> Save} or \inline{Ctrl-S}, then select
\inline{PlatformIO -> Build} or \inline{Ctrl-Alt-B} to build the program. The
compiler messages will appear, and then vanish after a couple of
seconds. Click \inline{PlatformIO -> Toggle Build Panel} or press F8
to display them again.

\subsection{Check for Build Errors}\label{check-for-build-errors}

You will see something like the following:

\begin{lstlisting}
Processing nanoatmega328 ...
------------------------------
Verbose mode can be enabled via `-v, --verbose` option
Collected 27 compatible libraries
Looking for dependencies...
Project does not have dependencies
Compiling .pioenvs/nanoatmega328/src/blink.o
...
Lots of "compiling ..." stuff here too.
...
Linking .pioenvs/nanoatmega328/firmware.elf
Building .pioenvs/nanoatmega328/firmware.hex
Calculating size .pioenvs/nanoatmega328/firmware.elf
AVR Memory Usage
----------------
Device: atmega328p
 
Program:     178 bytes (0.5% Full)
(.text + .data + .bootloader)
 
Data:          0 bytes (0.0% Full)
(.data + .bss + .noinit)
========================= [SUCCESS] Took 0.90 seconds ===============
\end{lstlisting}

Everything looks fine.

You may be wondering why we wrap the setting of \inline{F\_CPU} in the following manner:

\begin{lstlisting}[language=C,firstnumber=2,caption={Wrapping F\_CPU in AVRBlink.c}]
#ifndef F_CPU
#define F_CPU 16000000UL // 16 MHz clock speed
#endif
\end{lstlisting}

This is beacuse PlatformIO knows that an Arduino Nano with an AtMega328 runs with a 16
MHz crystal, so it defines the correct speed for us. This is defined in the file:

\inline{/home/norman/.platformio/packages/framework-arduinoavr/boards.txt}

Well, it is for me anyway, your mileage may vary, where a number of other details are defined
to make the system aware of the device, the speeds, baud rates for uploads etc.

If we are using something like Atmel Studio, then that might not set it correctly, so the correct setting would be applied there, as well as here.

\subsection{Gosh, Look How Small it
is!}\label{gosh-look-how-small-it-is}

Look at the last few lines of the compiler output text:

\begin{lstlisting}[firstnumber=14]
AVR Memory Usage
----------------
Device: atmega328p

Program:     178 bytes (0.5% Full)
(.text + .data + .bootloader)

Data:          0 bytes (0.0% Full)
(.data + .bss + .noinit)
\end{lstlisting}

This is telling you that the entire blink program takes up \emph{only}
178 bytes of Flash Memory and zero bytes of data space (in the Dynamic
RAM), on the Nano. 

I can even get it down to 158 bytes by replacing the code with the following:

\begin{lstlisting}[language=C,caption={AVRBlinkSmall.c}]
// This must also be done before anything else.
#ifndef F_CPU
#define F_CPU 16000000UL // 16 MHz clock speed
#endif

#include <avr/io.h>
#include <util/delay.h>

int main(void)
{
  // SETUP goes here.
  // Makes D13 aka PB5 (PORTB, Pin5) an output pin
  DDRB = (1 << PB5); 

  // And here is what Arduino calls loop.
  while(1) //infinite loop
  {
    // Toggle on D13 LED.
    PINB =  (1 << PB5);
    _delay_ms(1000);
  }
}
\end{lstlisting}

Compiling the above, results in this at the end:

\begin{lstlisting}[firstnumber=14]
AVR Memory Usage
----------------
Device: atmega328p
 
Program:     158 bytes (0.5% Full)
(.text + .data + .bootloader)
 
Data:          0 bytes (0.0% Full)
(.data + .bss + .noinit)
\end{lstlisting}

How much space does Blink take up when compiled in the Arduino IDE? Lets see:

\begin{lstlisting}
Sketch uses 928 bytes (3%) of program storage space. ....

Global variables use 9 bytes (0%) of dynamic memory, ....
\end{lstlisting}

We have a huge saving in such a tiny program. Just out of interest, I compiled the BareMinimum Arduino example, which looks like this:

\begin{lstlisting}[language=C]
void setup() {
  // put your setup code here, to run once:
}

void loop() {
  // put your main code here, to run repeatedly:
}
\end{lstlisting}

And the compiler told me that:

\begin{lstlisting}
Sketch uses 444 bytes (1%) of program storage space. ....

Global variables use 9 bytes (0%) of dynamic memory, ....
\end{lstlisting}

So, our blink code, written in AVR C with none of the Arduino wrappings,
takes up only 40\% of the space used in Flash Memory by a completely
empty\footnote{More on the reasons why elsewhere.} Arduino sketch!

\subsection{Upload to Our Board}\label{upload-to-our-board}

Uploading to the board can be either through a USB cable, as we do with the Arduino software, or, using an ISP programmer and the 6 pins on the Arduino board. Which is best?

By default, PlatformIO assumes that you will be using a USB cable, so the file \inline{platformio.ini} is already suitable, and when you come to upload, the desired port will (or should) be detected automatically.

If you are using an ISP programmer instead of the USB cable, then you must update the\\ \inline{platformio.ini} file to tell it which programmer you are using and the protocol. In addition, to upload firmware using an ISP programmer, you need to use \inline{target = program} instead \inline{target = upload} - which is the default. 

You will need to add \inline{target = program} to your \inline{platformio.ini} file. Don't forget to save it before attempting to upload/program!\footnote{Yes, I forgot!} When you next attempt an upload, you will see the text \inline{target: program} at the very top of the compile/upload listing when you press F8.


Here are a few configurations for various programmers, taken from the \href{http://docs.platformio.org/en/latest/platforms/atmelavr.html?highlight=usbtiny#upload-using-programmer}{PlatformIO docs}. Add the following lines to your \inline{platformio.ini} file for the corresponding programmer:

\begin{lstlisting}[caption={The \inline{platformio.ini} additions for `AVRISP' ISP Programmer}]
    upload_protocol = stk500v1
    upload_flags = -P$UPLOAD_PORT

    ; edit this line with valid upload port
    upload_port = SERIAL_PORT_HERE
\end{lstlisting}


\begin{lstlisting}[caption={The \inline{platformio.ini} additions for `AVRISP MkII' Programmer}]
    upload_protocol = stk500v2
    upload_flags = -Pusb
\end{lstlisting}


\begin{lstlisting}[caption={The \inline{platformio.ini} additions for `USBTinyISP' Programmer}]
    upload_protocol = usbtiny
\end{lstlisting}


\begin{lstlisting}[caption={The \inline{platformio.ini} additions for `ArduinoISP` Programmer}]
    upload_protocol = arduinoisp
\end{lstlisting}


\begin{lstlisting}[caption={The \inline{platformio.ini} additions for `USBasp' ISP Programmer}]
    upload_protocol = usbasp
    upload_flags = -Pusb
\end{lstlisting}


\begin{lstlisting}[caption={The \inline{platformio.ini} additions for `Parallel Port' ISP Programmer}]
    upload_protocol = dapa
    upload_flags = -F
\end{lstlisting}


\begin{lstlisting}[caption={The \inline{platformio.ini} additions for `Arduino as ISP` Programmer}]
    upload_protocol = stk500v1
    upload_flags = -P$UPLOAD_PORT -b$UPLOAD_SPEED

    ; edit these lines
    upload_port = SERIAL_PORT_HERE
    upload_speed = 19200
\end{lstlisting}

The latter one is used when you have a spare Arduino lying around doing nothing, and you have set it up to be used as an ISP programmer for AVR chips that don't have a bootloader, or simply because you want to! There will be more of this matter soon.

\begin{warning}
You should also be aware that if you program an Arduino using the ISP programmer, you cannot subsequently use the USB cable until you have burned the bootloader back into the chip. Ask me how I know! The steps to do this are simple:

\begin{itemize}
\item Connect your USB cable to the Arduino and to your computer.
\item Open the Arduino IDE.
\item Ensure that your board and port are correctly defined under \inline{tools -> board}.
\item Ensure that the programmer is set to AVRisp MK II (or AVR isp).
\item Select \inline{Tools -> Burn bootloader}.
\end{itemize}

After a few flashes of the built in LED, the bootloader will have been burned back into the chip. 
\end{warning}

The first time we do this with an AVR device, Arduino etc, we will need to
download the \inline{avrdude} tool. This is what we need to program
our device.

Select \inline{PlatformIO -> upload} and wait\ldots{}

Press F8 if the build information vanishes again.

You might see the following, on Linux:

\begin{lstlisting}
Warning! Please install `99-platformio-udev.rules` and check that your 
board's PID and VID are listed in the rules.
https://raw.githubusercontent.com/platformio/platformio/
develop/scripts/99-platformio-udev.rules
\end{lstlisting}

(The filename above is all on one line, I've split it to fit on the page.)

This is required to allow non-root users on a Linux system, the ability
to upload using the USB ports. 

On some systems you might need to add your 
user to the group \inline{dialout} or possibly \inline{plugdev} and 
restart your Atom session to pick up the new group. You can find the correct group
as follows:

\begin{lstlisting}[language=bash]
grep -i -e dialout -e plugdev /etc/group

dialout:x:18:
\end{lstlisting}

In my case, on Centos 7, it's the \inline{dialout} group, so:

\begin{lstlisting}[language=bash]
sudo usermod -a -G dialout $USER
\end{lstlisting}


\subsubsection{Install the Rules File}\label{install-the-rules-file}

Download and install the rules file as follows:

\begin{lstlisting}[language=bash]
cd /tmp

# The following is all on ONE LINE!
sudo wget https://raw.githubusercontent.com/platformio/platformio/
develop/scripts/99-platformio-udev.rules

sudo cp 99-platformio-udev.rules /etc/udev/rules.d/

sudo service udev restart
cd -
\end{lstlisting}

or, if that fails:

\begin{lstlisting}[language=bash]
sudo udevadm control --reload-rules
sudo udevadm trigger
cd -
\end{lstlisting}


