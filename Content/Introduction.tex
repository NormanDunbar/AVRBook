\section*{Introduction}\label{introduction}

This book is for those of us who have played with, and perhaps mastered, the \indexasis{Arduino} in its various forms, and who need now to move up to programming it - or home made clones on breadboards or stripboard (or even, home made PCBs) and so on - in AVR specific C code.

This need could be for many reasons, running out of space, too slow, lots of hidden things going on that are interfering with the project we have in mind etc. You might even have found that you can replace a full \indexasis{Arduino} with a small \indexasis{ATtiny85}, for example, and get exactly the same results, for a lot less outlay.

This book should help with that migration. It's not difficult, but it's not completely easy either. You will have to think a lot more at the hardware level, which the  \indexasis{Arduino} environment abstracts away for you, to make it simple.

There are three, that I know of anyway, levels of abstraction or interaction with the AVR micro controller in your Arduino:

\begin{itemize}
	\item The Arduino IDE \& language. This sits on top of the the avr-lib functions and is ideal for beginners.
	\item The avr-lib libraries. These sit on top of the underlying raw AVR code and are ideal for middling to skilled Arduinoists\footnote{Yes, I made up that word!}. 
	\item Raw AVR code. This is what the data sheets cover and explain in gory detail - which registers to set, which bits in said registers need to be set or unset, how and when they should be set/unset etc etc - it makes for a lot of bedtime reading!
\end{itemize}

This book deals, in the main, with the middle level of abstraction, the AVR-lib code and functions. Where necessary, however, I might have to dip into the lower level ``stuff'' when there's really no alternative.

Interrupts, timers and sleep modes will figure largely in what we do with AVR programming - we don't want to be wasting time waiting for someone to press a switch, or a sensor to register a reading, for example. And if there's nothing for the AVR to do while it waits around, we let it sleep! This saves power and could mean the difference between your project being able to run for months, if not years, on a couple of batteries, and having to be constantly plugged into the mains.

Just remember, \inline{delay()} is not your best friend when programming the  \indexasis{Arduino} or the AVR. Hopefully, you will find this out later when you read on.

\subsection*{Hardware Necessary}\label{hardware-necessary}

\begin{itemize}
\item
  An  \indexasis{Arduino}, perhaps. It's not essential, but you can use one to program bare AVR micro controllers - those you buy without a bootloader -   if you don't have a ISP programmer.
  
  A \emph{Nano} or \emph{Uno} etc are probably best. The \emph{Mega} is perhaps a tad overkill as an ISP programmer!
\item
  If you don't have a spare  \indexasis{Arduino} lying around, then you might want an \indexasis{ISP Programmer}. You don't \emph{need} one as the standard  \indexasis{Arduino} method of programming over the USB cable will still work.
  
  I got mine on eBay for around £3.28 including postage, from a seller named finetech007, in China. Other suppliers are available. The programmer allows you to save the 2Kb that an  \indexasis{Arduino} bootloader takes up, but obviously stops you programming the device using that bootloader, over the normal USB cable.

  Don't worry if you don't have an \indexasis{ISP Programmer}, as mentioned above, your existing  \indexasis{Arduino} can be used to program (other) AVR   micro-controllers. And if you don't have an  \indexasis{Arduino}, why not build your own? Grab hold of a \href{http://start.shrimping.it/project/protected/build.html}{Protected Shrimp} kit for around £10.00\footnote{Prices correct at the time of writing.}, or a more permanent (some soldering required) \href{http://start.shrimping.it/kit/stripboard.html}{Copper Shrimp} add on kit for £1.90 from those nice people at \href{http://start.shrimping.it//index.html}{ShrimpingIT}. I started with an Arduino Duemilanove which was bought for me as a gift, but I built my own Shrimp Kit just for the fun of it. I've built a few more since then!
\item
  A breadboard and assorted components, wires, etc. This is limited to the odd switch, LED and such like. Maybe a shift register or two will be handy - it depends on what you are expecting to achieve with the   AVR. (Or, what I decide to include in the book!)
\end{itemize}

\subsection*{Coming Up}\label{coming-up}

\textbf{To be continued ...}
